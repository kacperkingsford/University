\documentclass{article}

\title{Analiza Numeryczna (M) - Lista P0}
\date{12-10-2020}
\author{Kacper Kingsford}

\usepackage[T1]{fontenc}
\usepackage[utf8]{inputenc}
\usepackage{lmodern}

\usepackage{graphicx}
\graphicspath{ {/Desktop/programowanie/ANM/P0/doc/} }
\newtheorem{twierdzenie}{Twierdzenie}
\begin{document}

\maketitle
\pagenumbering{arabic}

\section{Dział}

Hello World!

\subsection{Poddział}

Hello World!

\subsection{Tabela zawierająca liczby}

\begin{center}
 \begin{tabular}{||c c c c||} 
 \hline
 $x$ & $t$ & $f(x)$ & $f(t)$ \\ [0.5ex] 
 \hline\hline
 1 & 6 & 87837 & 787 \\ 
 \hline
 2 & 7 & 78 & 5415 \\
 \hline
 3 & 545 & 778 & 7507 \\
 \hline
 4 & 545 & 18744 & 7560 \\
 \hline
 5 & 88 & 788 & 6344 \\ [1ex] 
 \hline
\end{tabular}
\end{center}

\subsection{Wykres}

Przykładowy wykres:
\begin{center}
\includegraphics[scale=0.4]{plot.png}
\end{center}


\subsection{Twierdzenie Taylora}

\begin{twierdzenie}
Załóżmy, że funkcja f ma pochodne wszystkich rzędów w pewnym otoczeniu U punktu $x_{0}$. Szereg:
\[ \sum_{n=0}^{\infty} \frac{f^{(n)}(x_0)}{n!} (x-x_0)^n = f(x_0) + \frac{f'(x_0)}{1!} (x-x_0) + \frac{f''(x_0)}{2!} (x-x_0)^2+ ... \]
nazywamy Szeregiem Taylora funkcji f w otoczeniu punktu $x_0$.
\end{twierdzenie}



\end{document}