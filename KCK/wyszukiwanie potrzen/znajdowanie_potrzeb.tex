\documentclass[12pt, a4paper, oneside]{report}
\usepackage[utf8]{inputenc}
\usepackage[T1]{fontenc}
\usepackage{setspace}
\usepackage{titlesec}
\usepackage[pdftex]{graphicx}     
\usepackage{float}
\usepackage{caption}
\usepackage{float}
\usepackage{xcolor}

\titleformat{\section}[block]{\Large\bfseries}{}{1em}{}
\usepackage[font={small, it}]{caption}


\title{\Huge \textbf{Znajdowanie potrzeb}\\
       \large \textbf{Recenzja pracy grupy Tabaczkowi}\\\,\\
       \Large Komunikacja człowiek-komputer 2020}
\date{\,\\2 listopada 2020}
\author{
\textbf{Grupa oceniająca}\\ Dominik Budzki\\ Kacper Kingsford\\ Adam Jarząbek\\
\textbf{Grupa oceniana}\\ Piotr Dobiech\\ Agata Kasprzak \\ Cezary Świtała\\

}

\setstretch{1.4}

\begin{document}

\begin{titlepage}
    \maketitle
    \thispagestyle{empty}
\end{titlepage}


\section*{Wstęp}
Wstęp 	jest zwięzły i przy tym dobrze opisuje problem. Jest napisany poprawnym i zrozumiałym językiem. Brak zastrzeżeń.

\section*{Opis badanych}
Opis ten dobrze przedstawia ankietowanych oraz ich potrzeby i problemy. Przedstawiono punkt widzenia zarówno właścicieli sklepów, jak i klienta. W tym punkcie nie znajdujemy żadnych nieprawidłowości.

\section*{Przebieg obserwacji}
Opis przebiegu obserwacji jest szczegółowy i skutecznie przedstawia problemy, z którymi zmagają się badani. Nie znajdujemy usterek ze strony merytorycznej, natomiast naszą uwagę zwróciło kilka błędów interpunkcyjnych: \begin{itemize}
    \item “Na dodatek nie ma porównania ofert konkurencyjnych sklepów, dlatego nie wie{\large\textbf{,}} w jaki sposób układać własne, aby były bardziej atrakcyjne.” ;
    \item “Niestety{\large\textbf{,}} zwrotu zakupów można dokonać jedynie przy okazaniu paragonu.”;
    \item “Wracając do domu{\large\textbf{,}} zauważył je na wystawce osiedlowego sklepiku w atrakcyjnej cenie.”;
    \item “W trakcie dokonywania zakupu zauważył, że jest w nim wszystko{\large\textbf{,}} czego potrzebował i żałował, że nie wybrał się tam wcześniej.” ;
\end{itemize}
Pomimo wymienionych wyżej błędów, opis jest czytelny i dobrze sformułowany. 


\section*{Ankieta}
Ankieta składa się z 12 pytań, z których 6 skierowanych jest do przedsiębiorców i 6 do klientów. Są one starannie dobrane, lecz brakuje w nich pytań zamkniętych, dzięki którym można by porównać odpowiedzi ankietowanych. Naszą uwagę zwrócił też układ tej części pracy - według nas czytelniej byłoby umieścić odpowiedzi do poszczególnych pytań bezpośrednio pod pytaniami, zamiast zbiorczo umieszczać wszystkie pytania i odpowiedzi osobno, co wymaga ciągłego przewijania strony podczas analizowania wyników ankiety. Zdziwiło nas też pewna niekonsekwencja - w przypadku pytań skierowanych do przedsiębiorców zwracacie się do ankietowanego per Pan/Pani, a w przypadku pytań do klientów - per ”ty”. Poza tym, ankieta jest dobrze przemyślana, a odpowiedzi są wartościowe.

\section*{Potrzeby}
Nie mamy zastrzeżeń do tej części pracy. Potrzeby są trafne i wynikają z wyników ankiety oraz obserwacji badanych.


\section*{Inspiracje}
Zaprzyjaźniona grupa podała 5 inspiracji. Wynikają one wprost z wymienionych wcześniej potrzeb, przy czym są dobrze rozwinięte i uzasadnione. Opisane pomysły uważamy za trafne i przemyślane. Brak zastrzeżeń. 


\section*{Ocena końcowa}
Podsumowując, grupa Tabaczkowych wykonała swoje zadanie ponad poprawnie. Wszystkie elementy pracy są rozwinięte i szczegółowe, a przy tym zwięzłe i czytelne. Być może pracę uatrakcyjniłoby dodanie zdjęć obrazujących omawiane zagadnienia. Pojawiły się pewne usterki techniczne i interpunkcyjne, lecz naszym zdaniem nie wpłynęły one negatywnie na ogólny odbiór pracy.




\end{document}