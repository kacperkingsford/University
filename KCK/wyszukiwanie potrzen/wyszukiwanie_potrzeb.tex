\documentclass[12pt, a4paper, oneside]{report}

\usepackage[utf8]{inputenc}
\usepackage[T1]{fontenc}
\usepackage{setspace}
\usepackage{titlesec}
\usepackage[pdftex]{graphicx}     
\usepackage{float}


\titleformat{\section}[block]{\Large\bfseries}{}{1em}{}
\usepackage[font={small, it}]{caption}


\title{\Huge \textbf{Znajdowanie potrzeb}\\
       \large \textbf{Raport z zadania}\\\,\\
       \Large Komunikacja człowiek-komputer 2020}
\date{\,\\2 listopada 2020}
\author{Dominik Budzki\\ Kacper Kingsford\\ Adam Jarząbek}

\setstretch{1.4}

\begin{document}

\begin{titlepage}
    \maketitle
    \thispagestyle{empty}
\end{titlepage}

\renewcommand*\thesubsection{\arabic{section}}

\section*{Charakterystyka problemu}

Pociągi to jeden z najpopularniejszych środków transportu w Polsce, dlatego ważne jest, aby cały system związany z tym środkiem lokomocyjnym działał sprawnie.
W dzisiejszych czasach powszechny dostęp do sieci internetowej sprawił, iż PKP musiały umożliwić Polakom zakup biletów przez internet.  Okazuje się jednak, że system proponowany przez PKP wzbudza mieszane uczucia wśród użytkowników. Postanowiliśmy zasięgnąć opinii przedstawicieli różnych grup społecznych.

\section*{Opis badania}

Poprosiliśmy badane osoby o zakup biletu na pociąg linii Katowice-Wrocław na stronie internetowej PKP Intercity. Po wykonaniu czynności wymaganych przed zakupem biletu zadaliśmy badanym kilka pytań związanych z procesem zakupu biletu przez stronę przewoźnika.

\section*{Opis osób badanych}

\begin{itemize}
    \item \textbf{osoba 1}: 53-letnia pani Anna, księgowa, pociągami nie jeździ od paru lat
    \item \textbf{osoba 2}: 20-letni student informatyki, w czasach przedpandemicznych jeździł pociągami dwa razy w tygodniu
    \item \textbf{osoba 3}: 74-letni pan Janusz, emerytowany inżynier, co jakiś czas jeździ pociągami z wnukami, ale bilety kupuje na dworcu.
\end{itemize}

\section*{Przebieg badania}

\begin{itemize}
    \item \textbf{1:} Pani Anna czuła się swobodnie, korzystając z komputera. Bez problemów wpisała nazwy stacji. Trudności przysporzyło jej ustawienie żądanej godziny odjazdu - pani Anna zmieniała ją ręcznie, korzystając z przycisków wyświetlających się na ekranie. Po wyszukaniu połączeń badana szybko znalazła przycisk “Kup bilet”. Gdy została przekierowana na stronę przewoźnika, jej oczom ponownie ukazało się wiele opcji przejazdu. Ponieważ pani Anna zapomniała, na którą godzinę przejazdu się zdecydowała, przez przypadek wybrała złą opcję. Postanowiła więc usunąć kartę ze stroną i powtórzyć cały proces. Problemem okazało się wybranie odpowiedniej ulgi. Następnie badana została poproszona o podanie danych logowania. Pani Anna nie zauważyła przycisku “kup bez rejestracji”, więc przeszła cały proces zakładania konta, z potwierdzeniem przez adres email włącznie. Gdy się zalogowała, okazało się, że proces kupowania biletu został przerwany. Pani Anna zrobiła więc wszystko od początku, dokańczając kupowanie biletu.
    \item \textbf{2:} Osoba ta korzysta z usług PKP raz w tygodniu, gdy wraca ze studiów na weekend do domu. Z uwagi na jej dobrą znajomość interfejsu oraz mechaniki strony, cały proces odbył się bez problemów.
    \item \textbf{3:}  Problem pojawił się już przy wyszukiwaniu odpowiedniego pociągu na rozkładzie jazdy. Po znalezieniu, z małą pomocą, odpowiedniego pociągu ankietowany został przekierowany na stronę przewoźnika. Problemem okazało się wybranie odpowiedniego miejsca. Czcionka i tło utrudniły badanemu wybranie przysługującej mu ulgi. Po odpowiednich ustawieniach pan Janusz, przechodząc dalej, został wyrzucony przez system. Musieliśmy zacząć cały proces od początku, co stało się powodem frustracji badanego. Dopiero po dłuższej chwili udało się przejść do etapu płatności za bilet.
\end{itemize}





\section*{Ankieta wraz z odpowiedziami}


\begin{itemize}
    \item\textbf{Czy uważasz proces kupowania biletu przez stronę PKP za przyjazny?}
    \begin{itemize}
        \item \textbf{Pani Anna:} Średnio.
        \item \textbf{Student:} Tak sobie.
        \item \textbf{Pan Janusz:} Nie.
    \end{itemize}

    \item\textbf{Wymień jedną rzecz, która sprawiła ci najwięcej problemów podczas kupowania biletu.}
    \begin{itemize}
        \item \textbf{Pani Anna:} Proces rejestracji.
        \item \textbf{Student:} Proces płatności.
        \item \textbf{Pan Janusz:} Wybranie ulgi.
    \end{itemize}

    \item\textbf{Jak często kupujesz bilety kolejowe?}
    \begin{itemize}
        \item \textbf{Pani Anna:} Bardzo rzadko.
        \item \textbf{Student:} Mniej więcej dwa razy w tygodniu.
        \item \textbf{Pan Janusz:} Parę razy w roku.
    \end{itemize}

    \item\textbf{Czy uważasz, że kupowanie biletów na dworcu jest wygodniejsze niż kupowanie ich przez stronę internetową?}
    \begin{itemize}
        \item \textbf{Pani Anna:} Tak.
        \item \textbf{Student:} Nie, kupowanie przez internet pozwala zaoszczędzić bardzo dużo czasu, szczególnie gdy się spieszę.
        \item \textbf{Pan Janusz:} Zdecydowanie tak.
    \end{itemize}

    \item\textbf{Czy twoim zdaniem strona PKP jest dobrze zaprojektowana?}
    \begin{itemize}
        \item \textbf{Pani Anna:} Nie spotkałam się z gorzej zaprojektowaną stroną internetową.
        \item \textbf{Student:} Mogło by być lepiej.
        \item \textbf{Pan Janusz:} Nie znam się w pełni na stronach internetowych, ale nie nazwałbym tej strony dobrze zaprojektowaną.
    \end{itemize}

    \item\textbf{Podaj jedną rzecz, którą byś zmienił.}
    \begin{itemize}
        \item \textbf{Pani Anna:} Wybieranie rodzaju biletu.
        \item \textbf{Student:} Ujednoliciłbym proces kupowania biletu, aby wszystko odbywało się na jednej stronie internetowej, w celu uniknięcia zbędnego przekierowywania między kilkoma innymi.
        \item \textbf{Pan Janusz:} Trudno wybrać jedną.
    \end{itemize}

    \item\textbf{Jak oceniasz swoje umiejętności korzystania z komputera?}
    \begin{itemize}
        \item \textbf{Pani Anna:} Myślę, że moje umiejętności są dość dobre, korzystam z komputera na co dzień w pracy.
        \item \textbf{Student:} Jako student informatyki nie mam najmniejszych problemów z obsługą komputera.
        \item \textbf{Pan Janusz:} Słabo, czasem sprawdzam wiadomości i pogodę. 
    \end{itemize}
\end{itemize}

\section*{Potrzeby}

Na podstawie odpowiedzi ankietowanych i własnych doświadczeń sformułowaliśmy dziesięć potrzeb :

\begin{enumerate}

    \item Zmiana interfejsu na bardziej przyjazny dla użytkownika.
    \item  Uproszczenie sposobu wyboru rodzaju biletu.
    \item  Ograniczenie procesu kupowania biletu do jednej domeny/strony.
    \item  Skrócenie procesu kupowania biletu.
     \item Dodanie możliwości wyboru miejsca.
     \item Usprawnienie systemu dokonywania płatności za bilet.
    \item  Zwiększenie limitu kupowanych biletów do liczby większej niż 4.
    \item  Wprowadzenie czatu z konsultantem.
    \item Proces rejestracji nie powinien przerywać procesu kupowania biletu.
     \item  Publikowanie informacji o utrudnieniach na etapie wyboru przejazdu.
\end{enumerate}



\section*{Innowacje, jakie można wykonać}

\begin{enumerate}
    \item Zmiana interfejsu na bardziej minimalistyczny.
    \item Umożliwienie wyboru konkretnego miejsca za pomocą mapy pociągu poprzez kliknięcie w dane miejsce.
    \item Dodanie “koszyka” z możliwością kupna więcej niż 4 biletów.
    \item Wprowadzenie czatu z konsultantem, który mógłby pomagać np. osobom starszym, niezaznajomionym z internetem.
    \item Umożliwienie korzystania ze strony w dwóch osobnych kartach przeglądarki.
\end{enumerate}



\end{document}

